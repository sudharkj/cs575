\documentclass[notitlepage]{report}

\usepackage{graphicx}
\usepackage{titling}
\usepackage{pgfplotstable}
\usepackage{longtable}

\setlength{\droptitle}{-10em}

\title{Vectorized Array Multiplication/Reduction\\ using SSE}
\author{Sudha Ravi Kumar Javvadi}

\begin{document}
	\maketitle
	
	\paragraph{Introduction}
	\paragraph{} This report documents an attempt to test array multiplication/reduction using SIMD and non-SIMD. First a simple comparison between SIMD and non-SIMD is presented without using multicore environment. Later, speedups are compared in multicore environments.
	
	\paragraph{Machine Details}
	\begin{itemize}
		\item{OS}: Ubuntu 18.04.04 LTS
		\item{Memory}: 15.5 GB
		\item{Processor}: Intel Core i5-9600K CPU @ 3.70GHz x 6
		\item{Graphics}: GeForce RTX 2070 SUPER/PCIe/SSE2
		\item{GNOME}: 3.28.2
		\item{OS type}: 64-bit
		\item{Disk}: 225.4 GB
	\end{itemize}

	\paragraph{Results}
	\paragraph{}
	\pgfplotstableset{
		begin table=\begin{longtable},
			end table=\end{longtable},
	}
	\pgfplotstabletypeset[
		col sep=comma,
		string type,
		columns/size/.style={column name=Size, column type={|c|}},
		columns/performance/.style={column name=Performance, column type={c}},
		columns/speedup/.style={column name=Speedup, column type={c|}},
		every head row/.style={before row={
				\caption{Table showing speedups with SIMD over non-SIMD}
				\label{tab:base}\\
				\hline},
			after row=\hline},
		every last row/.style={after row=\hline},
	]{../data/simulation.csv}
	An attempt to explain the above simulation result is done with the help of following graph.
	\begin{figure}[h!]
		\includegraphics [width=\linewidth] {../data/simulation.png}
		\caption{Graph showing speedups with SIMD over non-SIMD}
		\label{fig:base}
	\end{figure}
	\paragraph{} The graph showed there is atleast double the speedup with SIMD over non-SIMD code. But, somehow the speedup do not generalize for large arrays.
	
	\paragraph{Extra Credit}
	\paragraph{}
	\pgfplotstableset{
		begin table=\begin{longtable},
			end table=\end{longtable},
	}
	\pgfplotstabletypeset[
	col sep=comma,
	string type,
	columns/type/.style={column name=Type, column type={|l|}},
	columns/size/.style={column name=Size, column type={c}},
	columns/performance/.style={column name=Performance, column type={c}},
	columns/speedup/.style={column name=Speedup, column type={c|}},
	every head row/.style={before row={
			\caption{Table showing speedups with SIMD and cores over non-SIMD}
			\label{tab:ext}\\
			\hline},
		after row=\hline},
	every last row/.style={after row=\hline},
	]{../data/extra.csv}
	An attempt to explain the above simulation result is done with the help of following graph.
	\begin{figure}[h!]
		\includegraphics [width=\linewidth] {../data/extra.png}
		\caption{Graph showing speedups with SIMD and cores over non-SIMD}
		\label{fig:ext}
	\end{figure}
	\paragraph{} The graph showed that the speedup with SIMD with cores over non-SIMD is better compared to SIMD alone. But, somehow the speedup did not generalize for large arrays like earlier even with mutlicore environment.
	
	\paragraph{Code}
	\paragraph{} Unzip the file and run $./start.sh$ in the project folder to see the results and also populate the report.
\end{document}
